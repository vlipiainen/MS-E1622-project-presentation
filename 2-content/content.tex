%introduction-example

\section{Introduction}

\begin{frame}{Topic}
An overview of (part of) the article 
\[ \textbf{``Tensor decomposition and homotopy continuation''} \]
by Alessandra Bernardi, Noah S. Daleo, Jonathan D. Hauenstein, and Bernard Mourrain.
\end{frame}

\begin{frame}{Tensors and tensor rank}
  \begin{bee}[Tensor]
    A tensor can be defined as 1) a multidimensional array, 2) a multilinear map.
  \end{bee}

  \begin{bee}[Tensor rank]
    The rank of a tensor $T$ is the minimum number $r$ s.t. it can be written as a sum of $r$ simple tensors.
  \end{bee}

  \begin{bee}[Tensor decomposition]
     The simple tensors $T$ can be written as a sum of.
  \end{bee}
\end{frame}

\begin{frame}{The goal of the paper}
  \begin{bee}[Problem]
    Finding the rank \& decomposition of a tensor
  \end{bee}  
  \begin{bee}[Solution approach]
    \begin{enumerate}
      \item Generalise to a problem of testing membership in a join variety
      \item Calculate a pseudowitness set for that join variety
      \item Test membership in that join variety
    \end{enumerate}
  \end{bee}
\end{frame}


\begin{frame}{Outline}
  \tableofcontents
\end{frame}

\section{Background}
\subsection{Rank and join variety}
\begin{frame}{Rank of points in a variety}
  Let $X$ be a an irreducible and nondegenerate projective variety $X \subset \PP^n$ defined over $\CC$.

  \begin{bee}[Definition]
    The \emph{$X$-rank} of a point $P \in \CC^{n+1}$ is the minimum $r$ such that
    \[P = \sum_{i=1}^r x_i, \quad x_i \in \mathcal{C}(X),\]    
    where $\mathcal{C}(X)$ is the affine cone of $X$.
  \end{bee}
\end{frame}

\begin{frame}{Join variety}
  Let $\langle [x_1], \ldots, [x_r] \rangle$ denote the linear space spanned by $x_1, \ldots, x_r$.
  \begin{bee}[Definition]
    The \emph{constructible join variety} of irreducible and nondegenerate varieties $X_1, \ldots X_k$ is
      \[ J_0(X_1, \ldots X_k) = \bigcup_{[x_1] \in X_1 \ldots [x_k] \in X_k } \langle [x_1], \ldots, [x_r] \rangle \]
    and the \emph{join variety} is $J(X_1, \ldots, X_k) = \overline{J_0(X_1, \ldots, X_k)}$
  \end{bee}

  Setting $(\forall i) X_i = X$ we get the variety of all elements with rank at most k
\end{frame}

\subsection{Numerical algebraic geometry}

\begin{frame}{Witness sets}
  \begin{bee}[Informal definition]
    A witness set for a variety $X$ is a triple $\{f, \mathcal{L}, W\}$
    \begin{itemize}
      \item $f$ is is a set of polynomials defining $X$.*
      \item $\mathcal{L}$ is a general linear space that intersects with $X$.*
      \item $W = \mathcal{L} \cap X$
    \end{itemize}
  \end{bee}
\end{frame}

\begin{frame}{Homotopy continuation}
  \begin{bee}[Informal definition]
    Deforming a solution for one polynomial system is deformed to a solution for another polynomial system.
  \end{bee}
\end{frame}



\begin{frame}{Testing membership with witness sets}
  Given a point $p \in \CC^{N}$ and a variety $V \subset \CC^{N}$
  \begin{enumerate}
    \item Find a general linear space $\mathcal{L}_p \ni p$.*
    \item Track paths $\mathcal{L} \cap V \to \mathcal{L}_p \cap V$ along the homotopy
      \[V \cap (t \cdot \mathcal{L} + (1-t) \cdot \mathcal{L}_p)\]
    \item $p$ belongs to $V$ iff it is an endpoint for some element of $W$
  \end{enumerate}
\end{frame}



\section{The results}
\begin{frame}{Determining rank}
   \begin{bee}[Definition: Abstract join variety]
     \[
       \mathcal{J} = \bigl\{ ([P], x_1, \cdots, x_k) \mid x_i \in \mathcal{C}(X_i), P = \sum_{i=1}^k x_i \bigr\}
     \]
   \end{bee}
   If a given element has rank $\leq k$ the tuple $([P], \text{decompostion of P}) \in \mathcal{J}$ for $X_i = X$.
   
   For any candidate element, we can test membership in $J$ using a \emph{pseudowitness} set (which I don't have space to explain well) for $J$ and doing homotopy continuation along $\mathcal{J}$.
\end{frame}

\begin{frame}{Other results}
  \begin{itemize}
    \item Calculating a real decompostion
    \item Decompositions of generic elements using Newton homotopises
    \item Calculating upper bounds for rank and border rank using local numerical techniques and Newton homotopies.
  \end{itemize}
\end{frame}



\section*{}
\begin{frame}{Thank you}
  Thank you for your attention
\end{frame}

% \section{An example}
% \begin{frame}{Setup}
%   Lorem ipsum
% \end{frame}
% 
% \begin{frame}{Solution}
%   Lorem ipsum
% \end{frame}


% %real example
% \begin{frame}[plain]
%     \begin{bee}[Definición]
%         Sea $D$ un dominio de integridad. Una \textbf{\textit{función (o norma) euclídea}} sobre $D$ es cualquier función $v:D\backslash\{0\}\rightarrow\N$ tal que:
%             \begin{enumerate}[1)]
%                 \item Para todo $a,b\in D$, con $a\neq 0$, existen $q,r\in D$ tal que $a=bq+r$, donde $r=0$ o $v(r)<v(b)$.
%                 \item Para todo $a,b\in D$ no nulos, $v(a)\leq v(ab)$.
%             \end{enumerate}
%         Luego, un dominio de integridad $D$ es un \textbf{\textit{dominio euclídeo}} si existe una función euclídea en $D$.
%     \end{bee} 
% \end{frame}
% 
% %other examples
% \begin{frame}{Title}
%     \begin{bee}[]
%         Lorem ipsum dolor sit amet, consectetuer adipiscing elit.
%         Etiam lobortis facilisis sem. Nullam nec mi et neque
%         pharetra sollicitudin. Praesent imperdiet mi nec ante. Donec
%         ullamcorper, felis non sodales commodo, lectus velit ultrices
%         augue, a dignissim nibh lectus placerat pede. Vivamus nunc
%         nunc, molestie ut, ultricies vel, semper in, velit. Ut porttitor.
%     \end{bee}
% \end{frame}
% 
% \begin{frame}[plain]
%     \begin{columns}
%         \begin{column}{5cm}
%             \begin{bee}[]
%                 Lorem ipsum dolor sit amet, consectetuer adipiscing elit.
%                 Etiam lobortis facilisis sem. Nullam nec mi et neque
%                 pharetra sollicitudin. Praesent imperdiet mi nec ante. Donec
%                 ullamcorper, felis non sodales commodo, lectus velit ultrices
%                 augue, a dignissim nibh lectus placerat pede.
%             \end{bee}
%         \end{column}
%         \begin{column}{5cm}
%             \begin{bee}[note 1]
%                 Lorem ipsum dolor sit amet, consectetuer adipiscing elit.
%                 Etiam lobortis facilisis sem. 
%             \end{bee}
%             \begin{bee}[note 2]
%                 Lorem ipsum dolor sit amet, consectetuer adipiscing elit.
%                 Etiam lobortis facilisis sem. 
%             \end{bee}
%         \end{column}
%     \end{columns}
% \end{frame}
% 
% \begin{frame}{Title}
%     \begin{bee}[list]
%         \begin{enumerate}[$\bullet$]
%             \item element 1
%             \item element 2
%             \item element 3
%         \end{enumerate}
%     \end{bee}
%     \centering
%     \url{https://github.com/sdesch/minimalist-beamer-latex}
% \end{frame}
